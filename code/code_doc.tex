
\documentclass[12pt,preprint]{aastex}

\usepackage{amssymb,amsmath,mathrsfs,hyperref,datetime}

\setlength{\parindent}{0mm}
\newcommand{\tab}{\hspace*{2em}}

\begin{document}


\title{\tt HBSGSep} 

\author{Ross~Fadely, rfadely@haverford.edu}

\section{Introduction}

Found at \href{http://github.com/rossfadely/star-galaxy-classification}
{http://github.com/rossfadely/star-galaxy-classification} are {\tt C} routines 
which may be used to compile the code {\tt HBSGSep}.  Details about the purpose 
of the code, as well as the algorithm's concept and formalism can be found in 
the documents portion of the above URL.  It is recommended the user reads 
this document (Fadely, Hogg, and Willman (2012)) before proceeding.

As discussed in Fadely, Hogg, and Willman (2012), the routines provided here 
are a basic implementation of a hierarchical star-galaxy classification algorithm.  
In this light, the code provided here is meant to be conceptually clear --- it is 
by no means written for conciseness or optimal performance.  The hope, instead, 
is that the code may be useful for simple implementations, and inspirational for 
further algorithmic developments.  Please feel free to contact the author(s) with 
questions and comments.

Please do fork this code if you would like to make improvements.

\section{Compiling and Dependencies}

{\tt HBSGSep} makes use of two external packages - 

\tab\href{http://www.gnu.org/software/gsl/}{GNU Scientific Library} \\
\tab\href{http://heasarc.gsfc.nasa.gov/fitsio/}{CFITSIO}.

Before attempting to compile the routine, build the above packages (you may already 
have done so), and modify the makefile INCLUDES and LFLAGS to point to the path 
locations.  After doing so, typing {\tt make} at the terminal will compile the 
executable.  Before calling the executable (with no arguments), we first must modify 
a few variables to get started (Section \ref{sec:start}).

\section{Getting Started} 
\label{sec:start}

During runtime, {\tt HBSGSep} accepts and returns a number of FITS and ascii files.  
Before compiling, one must specify the path and names of these files as well as 
changing a few parameters.  These variables can be found at the beginning of 
{\tt HBSGSep.c}.  In this section we describe the variables which are \emph{required} 
to be modified.  Optional modifications are described in Section \ref{sec:optional}.

\subsection{Input Files}

{\tt filtersinput} -- Path to a file which specifies the disk locations of the throughput 
response curves for the filters of the photometric data.  The file format is each 
disk location, written as an ascii string, one per line.

{\tt galssedinput} -- Path to a file which specifies the disk locations of the galaxy SED 
templates.  The file format is each disk location, written as an ascii string, one per line.

{\tt starssedinput} -- Path to a file which specifies the disk locations of the star SED 
templates.  The file format is each disk location, written as an ascii string, one per line.

\subsection{Data file}

{\tt datafile} -- Path to the FITS table file containing the data to be classified.  The table 
must have at least 2$N_{filter}$ columns (where $N_{filter}$ is the number of filters), but may have 
more.  The first $N_{filter}$ columns should contain the magnitudes for each source, while 
the second $N_{filter}$ should contain the associated measurement uncertainties.  The order 
should be the same, e.g., $\{u,g,r,i,z,u_{err},g_{err},r_{err},i_{err},z_{err}\}$ should be the 
column order for $ugriz$ data.  The code can handle sources with non-detections or 
missing data.  For non-detections, the magnitude should be greater than 50 and the 
associated uncertainty should give the limiting magnitude of that filter, for that survey.  
If the data is missing, the magnitude should be negative.  If you prefer some other 
format, you need to modify {\tt calc\_datavals.c}.

\subsection{Output Files}

{\tt hypoutfile} -- Path to one-column FITS table file which will be written by the code 
after each improvement in the total log likelihood.  This file has $N_{star}+N_{gal}+2$
rows, where $N_{star}+N_{gal}$ are the number of star and galaxy templates, 
respectively.  The first $N_{star}$ values are the hyperparameter weights for the 
star templates, in the order they are given by {\tt starssedinput}.  These are followed by 
those for the galaxy templates.  Finally, the last two weights are the total prior probably 
that any source is any type of star or galaxy, respectively.

{\tt lnlikeratiooutfile} -- Path to one-column FITS table file which will be written by the code 
after optimization.  The file has $N_{data}$ lines, where each line is the loglikelihood ratio 
for a source, listed in the same order as the data file.  Loglikelihood ratios for each source 
are defined as the log of $\Omega$ in Eqn. 1 of Fadely, Hogg, and Willman (2012).

{\tt lnPtotfile} -- Path to one-column ascii file which is written after the specified 
number of iterations ({\tt writeiter}, see below).  The output file contains the values of the 
loglikelihood, one per line.  Useful for convergence monitoring, otherwise set to 
{\tt /dev/null}.

\subsection{Variables}

{\tt zmin, zmax, Nz}  -- Together these specify the redshift grid of the galaxy templates.  
Setting these gives the minimum and maximum redshift, and the number of gridpoints 
{\tt Nz}.  Fadely, Hogg, and Willman (2012) uses $\{0,4,51\}$ which gives a redshift 
step of 0.08.  We found this suitable for our purposes.

{\tt noisefudge} -- A few percent factor $\eta$ which is part of the noise model of the 
algorithm (see Appendix of Fadely, Hogg, and WIllman (2012)).  This value can be 
modified to find the maximum likelihood of the data.  Values between 0.04 and 0.08 
seem to be optimal for COSMOS data, but may be different for different survey, template, 
filter combinations.

{\tt Niter} -- The maximum number of loglikelihood calls made by the GSL optimizer.  By 
default, the optimizer is called twice, so 2\,{\tt Niter} iterations will be done.

{\tt tol} -- The tolerance for the simplex size of the downhill simplex optimizer. 

\section{Running tips and comments}

\subsection{Templates}

We provide a number of star and galaxy templates at 
\href{http://github.com/rossfadely/star-galaxy-classification}
{http://github.com/rossfadely/star-galaxy-classification}.  There are a few issues to 
consider when selecting the templates used in the classification.  First is the number 
of templates.  If the number of templates is very large, optimization can take a 
\emph{very} long time.  We have found $100-300$ star templates and $\sim 30$ 
galaxy templates (over 51 redshifts) works well.  Another thing to consider is whether 
the templates used span the color-color space of the data in question -- this may 
mean examining color-color plots beforehand.   Future versions of this algorithm will 
learn the number (and shape) of the templates from the data itself.

\subsection{Optimization}

Optimization in the code proceeds using a downhill simplex algorithm.  The optimization 
process will stop when either 2\,{\tt Niter} is reached or the simplex size falls below {\tt tol}.  
We have found greater success running for a fix number of iterations, rather than 
specifying an arbitrary tolerance (which changes as a function of the data, templates, etc.).  
For the optimization in Fadely, Hogg, and WiIlman (2012), {\tt Niter} $=50000$ was found 
to give a fairly good optimum.  

We note that for other versions of this algorithm we used a BFGS optimizer from GSL.  We 
found this to be only slightly faster than the simplex method, due to the cost of numerical 
derivatives.  Oddly, the BFGS did not seem to consistently give as good optimums as the 
one used here.

\subsection{Data}

We have found the results of {\tt HBSGSep} are not overly sensitive to the sample size 
of the data, except in extreme cases.  For large datasets, the code runs slower and may 
run into memory allocation issues depending on the user's machine.  For very small 
datasets, one runs the risk of not having data which spans the range of demographics 
and signal-to-noise of the entire dataset.  In Fadely, Hogg, and WIllman (2102), we find 
sample sizes between $3000-10000$ are adequate, but small sizes may be ok.  As a 
check, the user should make sure the optimum loglikelihood of any smaller dataset 
scales linearly with the size of larger sets.


\section{Optional features}
\label{sec:optional}

\subsection{Specifying initial hyperparameters}

The code allows you to specify input hyperparameters to start the optimization.  This is 
particularly useful if continuing from a previous optimization run.  Setting the variable 
{\tt usehypin} $=1$ will read in the parameters from 

{\tt hypinfile} -- Path to one column FITS table file, which specifies hyperparameter 
weights for star template, galaxy template, and the prior for star/galaxy.  If the optional 
analysis flag {\tt usehypin} $= 1$, this file will be read in as used as the starting point 
for the optimization.  This file is primary in place to allow continuation from previous 
optimization runs, and is not required if  {\tt usehypin} $= 0$.  See the description for 
{\tt hypoutfile} for a description of the file format.

\subsection{Calculate model magnitudes}

By setting {\tt calc\_model\_mags} $=1$, model magnitudes will be calculated for the 
templates.  One needs to specify:

{\tt starmodmagsfile} -- Path to output FITS table containing star model magnitudes.  
There will be $N_{filter}$ columns for each row, which are the magnitudes in the order 
given by {\tt filtersinput}.  Each row contains the values for the star templates, in the 
order given by {\tt starssedinput} .

{\tt galmodmagsfile} -- Path to output FITS table containing galaxy model magnitudes.  
This has the same column format as for {\tt starmodmagsfile}.  The rows, however, 
total $N_{gal}$ times {\tt Nz} in number.  The first {\tt Nz} rows are for the first template 
in {\tt galssedinput}, the next {\tt Nz} rows are for second first template, and so on.

Setting {\tt Niter} $=0$ will allow model magnitudes to be computed without the need 
to optimize hyperparameters.

\subsection{ML $\chi^2$ values}

One can calculate the maximum likelihood (minimum chi-squared) for each data 
point, for both star and galaxy models, by setting {\tt writeminchi2} $=1$.  This outputs
two files to

{\tt starchi2file} -- Path to output one-column FITS table.  The output file will contain 
the best chi-squared for \emph{any} star model for each source.  Row number 
corresponds to that of {\tt datafile}.

{\tt galchi2file} -- Same as {\tt starchi2file}, but for galaxy models.

Setting {\tt Niter} $=0$ will allow ML $\chi^2$ values to be computed without the need 
to optimize hyperparameters.

\subsection{Nuisance Parameters}

There are a few parameters which the algorithm uses but need not be changed.

{\tt fluxunitfactor} -- A scalar factor used in the template fitting to insure the values are 
kept well within machine precision. 

{\tt probfrac} -- The fraction of the ML star or galaxy model likelihood, which sets the 
tolerance for the ``sparse'' arrays propagated into the optimization.

{\tt P\_floor} -- The smallest probability assigned to anything.

{\tt weight\_fact} -- Sets initial step size of optimizer for total prior weights.

{\tt Ncstep} --  Number of steps used in marginalization over fit coefficient uncertainty.

{\tt writeiter} -- Number of iterations before update of current loglikelihood is printed 
to screen and to {\tt lnPtotfile}.

\end{document}

