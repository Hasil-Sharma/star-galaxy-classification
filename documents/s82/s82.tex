\documentclass[12pt]{article}

\newcommand{\project}[1]{\textsl{#1}}
\newcommand{\foreign}[1]{\textsl{#1}}
\newcommand{\etal}{\foreign{et~al.}}

\newcommand{\given}{\,|\,}
\newcommand{\dd}{\mathrm{d}}
\newcommand{\setofall}[1]{\left\{{#1}\right\}}
\renewcommand{\star}{\mathrm{star}}
\newcommand{\galaxy}{\mathrm{gal}}
\newcommand{\quasar}{\mathrm{qso}}
\newcommand{\mml}{\mathrm{MML}}

\title{Hierarchical probabilistic point-source classification for SDSS Stripe 82}
\author{some mix of Bochanski, Fadely, Hogg, Preston, Willman, and others}
\date{DRAFT VERSION 2012-10}

\begin{document}
\maketitle

\begin{abstract}
Ground-based optical studies of the Milky Way face the challenge that
at faint magnitudes, galaxies are not cleanly distinguishable from
stars on morphological grounds alone.  This problem is especially
severe at magnitudes at which galaxies enormously outnumber stars.
Fortunately, these surveys tend to be multi-band and the photometry
usually comes with reasonable uncertainty estimates.  Here we use an
unsupervised, template-based hierarchical Bayesian model to capitalize
on multi-band photometry to classify angularly compact sources into
star, galaxy, and quasar classes.  We apply the method to the
\project{Sloan Digital Sky Survey Stripe 82} imaging data set.  We
find XXX and YYY.  All our code and all of its output on the
\project{Stripe 82} data are available at ZZZ.
\end{abstract}

\section{Introduction}

Start with the ratios of stars, galaxies and quasars as a function of
apparent magnitude, color, and angular size cut.  This could be a
paper on its own!

Hogg to Beth: One way to think about this problem is as a data
imputation problem: Can we accurately predict the high-resolution
(say, \project{HST}-resolution) size measurement using the multi-band
low-resolution (ground-based) photometric measurements, in which the
source is unresolved?

Most approaches to this problem are supervised approaches.  These are
bad because test data (the new data of interest) are never
statistically like training data (previous-generation data no longer
of interest).

In previous work (Fadely \etal\ 2012) we have shown that an
unsupervised hierarchical Bayesian methodology outperforms the best
supervised classification methods in the realistic situations in which
we find ourselves: We have a good noise model for our data, but we
don't have training data at the same depth or signal-to-noise as the
test data (data of interest).

\section{Method}

\begin{eqnarray}
p(d_i\given j,C_{ij},I) &=& A_i\,\exp(-\frac{1}{2}\,\chi^2_{ij})
\\
p(d_i\given j,I) &=& \int p(d_i\given j,C_{ij},I)\,p(C_{ij}\given I)\,\dd C_{ij}
\\
p(d_i\given \star,I) &=& \sum_j p(d_i\given j,I)\,P(j\given I)
\\
p(d_i\given I) &=& \sum_{q\in\setofall{\star,\galaxy,\quasar}} p(d_i\given q,I)\,P(q\given I)
\\
p(\setofall{d_i}_i\given I) &=& \prod_i p(d_i\given I)
\\
I_{\mml} &\leftarrow& \arg\max_I p(d_i\given I)
\\
p(\star\given d_i,I_{\mml}) &=& \frac{1}{Z_i}\,p(d_i\given \star,I_{\mml})\,P(\star\given I_{\mml})
\\
Z_i &\equiv& p(d_i\given I_{\mml})
\quad ,
\end{eqnarray}
where $I$ represents all the assumptions and hyper-parameters and
$I_{\mml}$ is the maximum-marginalized likelihood values for all those
assumptions and hyperparameters (that we permitted to vary).  This
method is called ``empirical Bayes'', apparently.

Beth question: Are the posterior probabilities true probabilities?
Hogg answer: Yes, but conditional on some hella unrealistic
assumptions.

\section{Data}

Summary of \project{SDSS Stripe 82} data and how we got it.  Don't
forget the errors.

Discussion of \project{SDSS} star--galaxy separators in common use.

\section{Results}

How are we going to test the results?  One option is to go to sources
covered by \project{HST} pointings and plot the \project{HST}-measured
size against each of several star--galaxy separation variables,
including likelihood ratios, posterior probabilities, SVM boundary
distances, psf-minus-model magnitudes, and the like.

Bochanski went through and grabbed a bunch of Sextractor source lists
from ACS, WFC3 and WFPC2 images from the HLA.

VVDS and PRIMUS have unbiased spectroscopic databases that would be
useful for testing the star--galaxy separators. 

Another more speculative is to see if we can see significance of some
\project{Stripe 82} stellar structure rise as we apply our method.

Another is to see if we predict photometric bands or colors \emph{not}
used in the HB inference.  That would be a ``posterior predictive
check'' which is not \emph{directly} relevant, but should correlate
with performance.

\section{Discussion}

What worked, what didn't?

What approximations did we make and how are they likely to be hurting
us?

What advice do we have for just-starting projects like \project{Dark
  Energy Survey}?  And of course \project{LSST}.

\end{document}
